\section{Conclusion} \label{sec:conclusion}

In this work, we presented our implementation of a decentralized car marketplace that is built upon the Ethereum blockchain. With this, we intend to demonstrate how the architecture of a state-of-the art web application can be combined with the Ethereum blockchain, yielding a hybrid approach to a blockchain powered application. While we still rely on a centralized front-end, the important business logic is running decentralized in form of smart contracts on the Ethereum network. Further, we believe that this hybrid approach helps developers, as well as stakeholders and end users to build up confidence with blockchain technology in the context of smart mobility. To name one example: the use of Ethereum as our application's backbone is almost transparent to the end user and any interaction with the blockchain is designed as intuitive and simple as possible. At the same time our implementation manages to provide the desired benefits of blockchain technology.

While our implementation already provides a rich feature set that covers all the core functionality of a car marketplace platform, we identify some task that could be tackled in future work. First, we made some assumptions in Section~\ref{concept:assumptions}. It is still open to evaluate these assumptions from a business perspective. E.g, would car dealers be willing to join such an ecosystem, what are possible hindrances?
Secondly, in this paper we focused on the software development part and not on legal issues. Thus, the legal compliance needs to be assessed. Lastly, Ethereum is just one of many blockchain technologies and many new ones have since been proposed that claim to solve issues of existing ones~\cite{NeoWhitePaper, EOSWhitePaper}. Therefore, it might be of value to compare, in the context of this implementation, Ethereum with newly proposed platforms in order to assess whether they indeed manage to solve these issues.
