\section{Introduction}
In the course of this work we evaluate an example implementation of a car marketplace based on blockchain technology. We start this introduction with context on how blockchain has become such a popular topic in recent years. Afterwards, we will discuss the purpose of this work as well as our objectives. We will also introduce some other similar applications that already made efforts to leverage the advantages of blockchains.

\subsection{Motivation}
The progressive technological development of the last decades has led to the emergence of massive amounts of data that were hardly imaginable in the past. More and more information is being digitized daily, including personal data. The secure management of this data, especially private data, is a challenging task and has been a source of concern for some time. Most people cannot conclusively determine how their data is handled once it has been handed over, leading to insecurity and discontent. Regardless of which management solution a third party actually uses to handle user data, users will always need a certain degree of trust before entrusting their sensitive data to someone else. One example in which trust in data is particularly important are car marketplaces. Customers need to be certain that vehicle information provided by a platform has not been altered to mislead them into buying an improperly advertised vehicle.

In 2008 Satoshi Nakamoto \cite{nakamoto2008bitcoin} first presented Bitcoin, an approach for a peer-to-peer electronic cash system using a public transaction ledger, also called blockchain. With this ledger, he is able to provide secure transactions between two parties without requiring any trust. This idea directly addresses the data privacy concerns that many people currently share. As a result, blockchain technology has sparked great interest as it could potentially improve the current state of data privacy in our society. Other popular blockchain technologies such as Ethereum \cite{Ethereum} are already working on advancing this idea and enable developers to leverage blockchains in both new as well as existing applications.

For this reason, it is of great interest to dive deeper into the theoretical concepts behind blockchains to see how they address the above-mentioned issues. At the same time, it is important to put this idea into a practical context, such as a car marketplace, and evaluate its potential impact.

\subsection{Problem Statement}
Despite the great interest and potential of blockchains, this rather new technology has not yet been widely adopted. On the one hand, people new to this type of technology may not have the time or the desire to work through the multitude of available resources to attain a level of confidence to use blockchain-based services themselves. This issue may be amplified by the large amount of information revolving around new blockchain-based technologies that is circulating the media on a daily basis. On the other hand, companies may not have yet the incentive, that would justify spending time and resources to deploy blockchains internally or in their products.

\subsection{Contribution}
In the course of this work we evaluate the application of blockchains for software development presenting a sample implementation of a blockchain-based car marketplace. In our application we utilize the open-source blockchain platform Ethereum \cite{Ethereum}. By using smart contracts, we can manage data in a secure manner. Adding this component to our otherwise more traditional architecture we are able to leverage the advantages of blockchains while building upon already existing ideas.

This work provides both a theoretical introduction to the topic of blockchains as well as insights to the implementation of applications based on blockchain. This may assist readers to further pursuit the subject of blockchains both theoritically and practically. Additionally we discuss the potential advantages of our application over more traditional approaches. As a result of this work, readers may be able to better reflect on their own ideas.

\subsection{Related Work}
Many people have already made great efforts to use blockchains and to extend or implement new applications based on them. One of these applications is LBRY \cite{lbry}, a digital marketplace that is not controlled by a central authority but by its participants instead. Users providing content on this platform are directly associated with that content and have control over it. LBRY uses blockchain technology to store and lookup decentralized metadata. An user can store his data in this blockchain by bidding on LBRY names. The owner of such a name is in control over its content as well as its access rights. Another platform that involves cars and uses blockchain technology is carVertical \cite{carVertical} which allegedly enables users to perform a decentralized vehicle check using blockchain. Among other things, this may be used to browse the history of a vehicle or to check whether a vehicle has been stolen. BitCar \cite{BitCar} is another blockchain-based platform that involves car assets. On this platform, users can allegedly trade fractional ownerships of cars and possibly aquire complete ownership of luxury cars.

\subsection{Outline}
The rest of this paper is divided into seven sections. In Section~\ref{sec:foundations} we present the theoretical foundations to blockchains, which will help the reader to understand the core concepts of the topic. In Section~\ref{sec:concept} we discuss the general concepts of our example implementation and describe the requirements and assumptions. Based on these concepts, we describe our implementation and explore the technical aspects  in Section~\ref{sec:impl}. This includes the smart contracts, data structures and data flow in our application. The selling points of our prototype are discussed in Section~\ref{sec:selling_points}.
Finally, Section~\ref{sec:conclusion} summarizes our work, mentions some of the remaining challanges and provides a future perspective on the topic.
